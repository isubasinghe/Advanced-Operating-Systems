
\documentclass[10pt]{beamer}

\usetheme[progressbar=frametitle]{metropolis}
\usepackage{appendixnumberbeamer}

\usepackage{booktabs}
\usepackage[scale=2]{ccicons}

\usepackage{pgfplots}
\usepgfplotslibrary{dateplot}
\usepackage[edges]{forest}
\usetikzlibrary{shapes.geometric,arrows.meta}
\usepackage{xspace}
\usepackage{graphicx}
\graphicspath{ {./images/} }

\newcommand{\themename}{\textbf{\textsc{metropolis}}\xspace}

\title{Advanced Operating Systems}
\subtitle{An exploration of alternative OS designs}
\date{\today}
\author{Isitha Subasinghe}
\institute{University of Melbourne}
\begin{document}

\maketitle

\begin{frame}{Table of contents}
  \setbeamertemplate{section in toc}[sections numbered]
  \tableofcontents[hideallsubsections]
\end{frame}

\section[Introduction]{Introduction}

\begin{frame}[fragile]{Introduction}

\textbf{What is an Operating System?}

Differential Dataflow is a generalisation of Incremental Computation. It allows for \textcolor{blue}{incrementally} updating the results of \textcolor{blue}{structured computations} over \textcolor{blue}{multisets}.

\end{frame}


\begin{frame}[fragile]{Background}
  Related Work
  \begin{itemize}
    \item{Mach}
    \item{L3}
    \item{L4}
    \item{sel4}
    \item{Barrelfish}
  \end{itemize}
\end{frame}

\section[Allocators]{Allocators}
\begin{frame}{Physical Memory Management}
  Here
\end{frame}

\subsection[Fragmentation]{Fragmentation}
\begin{frame}{Fragmentation}
  Fragmentation
\end{frame}

\subsection[List Based Allocators]{List Based Allocators}
\begin{frame}{List Based Allocators}
  FIRST FIT 
  BEST FIT 
  NEXT FIT
  DESCRIBE ALT - BUDDY
\end{frame}

\section[Virtual Memory]{Virtual Memory}
\begin{frame}{Virtual Memory}
  Virtual Memory 
  Origins of segmentation and "segmentation fault" is kinda interesting
\end{frame}

\section[Permissions]{Permissions}
\begin{frame}{Permissions}
  Permissions
\end{frame}

\begin{frame}{Access Control Lists}
  ACL
\end{frame}

\begin{frame}{Confused Deputy}
\end{frame}

\begin{frame}{Capabilities}
  WHY NOT CAPS?
\end{frame}






\begin{frame}[allowframebreaks]{References}

  \bibliography{demo}
  \bibliographystyle{abbrv}

\end{frame}

\end{document}
